\chapter{Conclusions}
\label{ch:conclusions}
In this dissertation we developed a series of deep learning based approaches to make human identification scalable towards real-world data and applications. From the data perspective, on one hand we addressed the challenge of lacking enough supervised data. A deep learning framework is proposed to utilize semi-supervised noisy-labeled data, which is much cheaper and efficient to collect. On the other hand, we tackled the problem of using multiple existing small datasets that each has its own data bias. A joint single task learning algorithm and domain guided dropout technique are developed to handle the domain bias explicitly in a single model. From the application perspective, we addressed on the more realistic problem setting, proposed a unique deep learning framework for simultaneous person detection and identification. A novel Online Instance Matching loss function is also proposed to learn identification features more effectively.

Despite our efforts toward scalable human identification, the problem itself still poses many challenges. We elaborated in Section~\ref{sec:human-id-tech-roadmap} a technical roadmap for human identification. More research should be conducted to answer the questions of how to learn good features from a single image, how to compare a pair of feature maps, and how to structure a set of features. More application scenarios and problem settings are also worth investigation, including video-based human identification, as well as connecting images and text descriptions.
