%%--------------------------------------------------------------------------
\chapter*{摘要}
\addcontentsline{toc}{chapter}{Abstract in Chinese}
\markboth{\uppercase{Abstract}}{\uppercase{Abstract}}
%%--------------------------------------------------------------------------
人物识别是计算机视觉领域非常重要的一类基础问题,其主要研究的是如何从一个图片或者视频集中找出一个目标人物。人物识别广泛应用于实际场景,是很多产品的重要组件,例如各类手机相册应用、自动驾驶系统、家庭安全防范系统、以及智能监控摄像头等等。

依托于深度学习技术的发展和大型带标注数据集的出现,深度神经网络目前已经可以在照片中识别上千种物体。但是人物识别仍然是一个很有挑战性的问题,因为该领域缺乏一个足够大规模的、带标注的数据集以训练有效的深度神经网络。与此同时,已有的小规模数据集通常收集自不同的场景,其图片各有偏颇,导致很难训练一个模型使其在所有的数据集上同时有效。另外,已有的研究工作多数集中于较为简单的问题设置,和真实的应用场景存在较大偏差,使得很难将研究直接用于实际产品。

因此,在本文中我们将从三个角度解决这些问题,从而使人物识别技术扩展到真实的数据和应用上面。首先,我们提出使用带噪声、易收集的数据来代替精标注、难收集的数据来训练模型。我们还从互联网上收集了一个大规模的、含有错误标注的衣物数据集,并用我们提出了的深度学习模型从此类带噪声的数据中学习有用的服装特征,从而帮助人物识别。其次,我们针对多个各有偏颇的数据集,提出了一个联合此类数据训练的深度神经网络的方法,从而使其可以有效地同时利用这些不同研究机构和个人收集的数据。最后,我们着手解决一个更贴近实际应用的问题设置,即从完整的场景图片中找出并识别目标人物。我们设计了一个统一的深度神经网络,同时进行人物检测和识别,并且提出了一个新的目标函数用以更加高效地训练整个网络。
